\documentclass{article}

% Packages
\usepackage{titling}
\usepackage{geometry}
\usepackage{fancyhdr}
\usepackage{enumitem}
\usepackage{algpseudocode}
\usepackage{algorithm}
\usepackage{setspace}
\usepackage{amsfonts}
\usepackage{graphicx}
\usepackage{amsmath}
\usepackage{cite}
\usepackage{pdfpages}

% Page setup
\geometry{a4paper, margin=1in}
\pagestyle{fancy}
\fancyhf{}
\renewcommand{\headrulewidth}{0pt}

% Metadata
\newcommand{\psetauthor}[1]{\author{#1}}
\setlength{\parindent}{0pt}

\newenvironment{problem}[1]{
  \clearpage
  \section*{Problem #1}
}{
  \clearpage
}

\doublespacing
\begin{document}

% Title Page
\title{Report on paper:
 \\ \textit{Vertex Connectivity in Poly-logarithmic Max-flows}
 \\ \small{Presentation outline is in Appendix}}
\psetauthor{Hansheng Liu (hl58)}
\date{\today}
\maketitle
\clearpage

\begin{section}{Introduction}
The \textbf{vertex connectivity} of a graph $G$ is the minimum $k$ s.t. for some $S \subseteq V(G)$ with $|S| = k$, graph $G - S$ either is disconnected or has at most one vertex. Connectivity is one of the most important and well-studies problems in graph theory. It can also be applied in many fields such as Computer Science and Tranportation. However, the algorithm to compute the vertex connectivity for a given graph is not an easy problem. For a long time, the best algorithm people have requires a running time of $\tilde{O}(mn)$ \cite{vertex_connectivity_henzinger}. In 2021, Saranurak et al. proposed a new algorithm that reduced the vertec connectivity problem to polylog maxflow instances \cite{vertex_connectivity}. In 2022, Chen et al. gave an algorithm that compute the maxflow in $m^{1 + O(1)}$ time \cite{max_flow_min_cost}, which can be used as a black box and make the vertex connectivity be computed in $\tilde{O}(m)$. In this report, we will focus on the idea of how to reduce vertex connectivity problems into polylog maxflow instances. 
\end{section}

\begin{section}{The main algorithm and its analysis}
  \begin{subsection}{Basic settings and assumptions}
    Let $|V(G)| = n$, $|E(G)| = m$. Let $S$ where $|S| = k$ denote the smallest cut. There will be at least two connected components (if $G$ is not $K_n$) in $G - S$. Let the smallest one be $L$, others be $R$. So $|L| \leq |R|$. Let $V_{low} = \{v | d(v) \leq 8k\}$. $L_{low} = L \cap V_{low}$, and same for $S_{low}, R_{low}$. 

    \textbf{Assumption 1}: We already know $k, |L|, |S_{low}|$. We will prove later that we don't need exact true value to make the algorithm work. Actually, we just need the value to be inside a range between the exact true value, which can be done by binary search.

    \textbf{Assumption 2}: $|R| = \Omega(n)$. This means $G$ does not have a minimum vertex cut that covers most of the verticies. Examples include complete graph, or some highly connected graph. 

    \textbf{Assumption 3}: $m \leq nk$. This clearly not applies to many graphs. However, we can achieve this by computing a \textit{k-connectivity certificate} \cite{nagamochi_ibaraki} for $G$, which results in this. 

    We divide the algorithm into three cases depends on $L$ and $S_{low}$. 
  \end{subsection}

  \begin{subsection}{Case 1: Large $L$}
    In this case, we assume $|L| > \frac{k}{polylog(n)}$. We independently sample each vertecx with probability $\frac1k$, let the sampled set be $T$. 

    \textbf{Claim 1}: With some probability, $T$ contains exactly one vertex from $L$, no vertex from $S$, and all others (at least one) from $R$. 

    Proof: $$\mathbb{P}[|L \cap T| = 1] \cdot \mathbb{P}[|S \cap T| = 0] \cdot \mathbb{P}[|R \cap T| \geq 1]$$
    $$ \geq c \cdot \mathbb{P}[|L \cap T| = 1] \cdot \mathbb{P}[|S \cap T| = 0]$$
    $$ \geq c \cdot (\frac{|L|}{k} \cdot (1 - \frac1k)^{|L| - 1}) \cdot (1 - \frac1k)^{|S|}$$
    $$ \geq c \cdot \frac{|L|}{k} \cdot (1 - \frac1k)^{|L| + |S| - 1}$$
    $$ \geq \Omega(\frac{1}{\log n})$$

    We can repeat polylog(n) times to increase probability. 

    After we obtained $T$, we can use modified \textit{Isolating Cuts} \cite{chekuri_quanrud} to compute the final results. We will skip the detail of this part as it will distract our main focus. 
  \end{subsection}

  \begin{subsection}{Case 2: Small $L$ with small $S_{low}$}
    In this case, we assume $|L| < \frac{k}{polylog(n)}$ and $S_{low} < |L| \cdot polylog(n)$. 

    \textbf{Claim 2}: $|L \cup S| \leq 2k$ 

    Proof: $|L| < \frac{k}{polylog(n)}$, $|S| = k$. 

    \textbf{Claim 3}: $L = L_{low} \subseteq V_{low}$

    Proof: For every vertex $x$ in $L$, all its neighbours must in $L \cup S$. So $d(x) \leq L + k \leq 2k < 8k$. 

    \textbf{Claim 4}: $|R_{low}| \geq |L_{low}|$

    Proof: If $k \geq \frac{n}{8}$, then all verticies are in $V_{low}$, the claim is trivial. Therefore, assume $k < \frac{n}{8}$. Notice that 
    $$ 8k \cdot |V - V_{low}| \leq \sum_v d(v) = 2m \leq 2nk$$
    Therefore, we have $|V - V_{low}| \leq \frac{n}{4}$, so $|V_{low}| \geq \frac{3}{4}n$. Also, by claim 2, we have $|R| \geq \frac{3}{4}n$. Therefore, $|R_{low}| \geq (1 - (1 - \frac34) - (1 - \frac34))n \geq \frac{n}{2} \geq |L_{low}|$. 

    We can independently sample vertex from $V_{low}$ with probability $\frac{1}{|L| \cdot polylog(n)}$ and get in to the same situation as in case 1. The detailed analysis is similar. 
    
  \end{subsection}

  \begin{subsection}{Case 3: Small $L$ with large $S_{low}$}
    In this case, we assume $|L| < \frac{k}{polylog(n)}$ and $S_{low} \geq |L| \cdot polylog(n)$. In this case, the sampling method we used in previous cases will not work as there is not a way to sample a desired vertex set with high probability. Therefore, we need to do something different. 

    \textbf{A quick question}: What information do we need to compute the vertex connectivity? 

    Answer: A vertex from $L$ and at least one vertex from $R$. Then, the vertex connectivity can easily be computed by a maxflow. 

    Notice that if we do random sample from $V$, have verticies from $R$ is with very high probability, but have a vertex from $L$ is with very low probability. Actually, you need to sample at least $\tilde{O}(\frac{n}{|L|})$ to make sure there is at least one vertex from $L$ with high probability, call this set $X$. By constant number of sampling, we can assume we have at least one vertex from $R$, call it $r$.

    \textbf{Silly algorithm}: For each vertex $x$ in $X$, compute the maxflow from $x$ to $r$. The minimum cut of the maxflow is the vertex connectivity.

    This algorithm works, but it is too slow. The runtime is $\tilde{O}(\frac{n}{|L|} \cdot m)$. We need to improve it.

    \textbf{Improvement idea}: If we are able to reduce the number of edges during each maxflow computation to $\tilde{O}(k|L|)$, then we can reduce the total runtime to $\tilde{O}(\frac{n}{|L|} \cdot k|L|) = \tilde{O}(nk) = \tilde{O}(m)$. (This $m$ is the number of edges before we compute the k-connectivity certificate)

    \textbf{Lemma 1}: For each vertex $x$ in $X$, we can compute a new graph which keeps the vertex connectivity and has at most $\tilde{O}(k|L|)$ edges. We call this graph a \textit{kernel}. 

    We now elaborate on how to construct the kernel. Assume we already have vertex $x \in L$, which is from $X$. Independently sample each vertex from $V$ with probability $\frac{1}{|L|}$ into set $T$. Let $N[x] = N(x) \cup \{x\}$ and $T_x = T - N[x]$.

    \textbf{Claim 5}: $|(L \cup S) - N[x]| < |L|$

    Proof: Notice that $|L \cup S| \leq |L| + k$. Also, $|N[x]| = |N(x)| + 1 \geq k + 1 > k$. Therefore, $|(L \cup S) - N[x]| < |L| + k - k = |L|$.
    
    If no verticies we sampled are from $|(L \cup S) - N[x]|$, then $T_x \subseteq R$. By claim 5, if we sample with probability $\frac{1}{|L|}$, we will have $T_x \subseteq R$ with high probability.

    After this, we contract $T_x$ into a single vertex $t_x$. Notice that if $x \in L$ and $T_x \subseteq R$, then a $(x, t_x)$ maxflow will give us the vertex connectivity of $G$. 

    \textbf{Observations}:
    \begin{enumerate}
      \item $\forall \ v \in N(x) \cap N(t_x)$, it must be in $S$. 
      \item By Mengers theorem, there exists $k$ vertex disjoint paths between $x$ and $t_x$. Notice that if any path include more than one neighbour of $x$, then the path can be changed to directly connect to $x$ without affect other paths. Same for $t_x$. Therefore, we can have each path only contains one neighbour of $x$ and one neighbour of $t_x$. 
    \end{enumerate}

    By these observations, we can do following removals \footnote{If reader find the following removal procedures hard to understand, there is a graph illustration in the appendix.}:
    \begin{enumerate}
      \item Remove all $v \in N(x) \cap N(t_x)$ and add back later. 
      \item Remove all internal edges of $N(x)$ and $N(t_x)$. 
      \item After first two, remove all $v \in N(T_x)$ s.t. $t_x$ is the only neighbour of $v$. This happens if $t \in N(v) \subseteq N[t]$. 
    \end{enumerate}

    \textbf{Claim 6}: After these removals, the graph has $\tilde{O}(k|L|)$ edges, which is the kernel we want.

    Proof: Let $N_x$ be the remaining neighbours of $x$, $N_t$ be the remaining neighbours of $t_x$, $F'$ be all verticies left. There are three kinds of edges remain:
    \begin{enumerate}
      \item [$E1$]: $N_x \times (F' \cup N_t)$ 
      \item [$E2$]: $F' \times (F' \cup N_t)$
      \item [$E3$]: incident with $x$ and $t_x$
    \end{enumerate}
    We first prove $|E1 \cup E2| = \tilde{O}(k|L|)$ whp. We prove by showing the endpoints and their degree are bounded. 

    \textbf{Claim 7}: $\forall \ v \in N_x \cup F', d_{G - N_x} (v) \leq \tilde{O}(|L|)$ whp. 

    Proof: We prove a stronger claim $\forall \ v \in N[x] \cup F', d_{G - N[x]} (v) \leq 8k$ whp. Assume not, then $d_{G - N[x]} (v) > |L| \cdot polylog(n)$, then at least one of $v$'s neighbour will be sampled into $T_x$ whp, which makes $v \in N(T_x) = N_t$, thus $v \notin N_x \cup F'$. Contradiction. 

    \textbf{Claim 8}: $|N_x \cup F'| = O(k)$ 

    Proof: Note that $|N_x| \leq |L| + k < 2k$, so we just need to show $|F'| = O(k)$. We go back to the origin graph $G$. 

    \textbf{Claim 8.1}: $|N(x)| < k + |L|$

    Proof: $|N(x)| \leq |L| \cup |S|$. 

    \textbf{Claim 8.2}: $|S_{low} - N(x)| \leq |L|$

    Proof: direct result of claim 5. 

    \textbf{Claim 8.3}: $\forall \ v \in F'$, whp, at least $k - |L| \cdot polylog(n)$ neighbours of $v$ are in $N(x)$.

    Proof: derect result of claim 7. 

    By claim 8.1, 8.2, 8.3, we can compute the number of neighbours of $v \in F'$ in $S_{low}$, which is 
    $$ \geq \text{Total number of neighbours in $N[x]$ - verticies in ($N[x] - S_{low}$)}\footnote{If reader find this hard to follow, there is a graph illutration in appendix.}$$
    $$ \geq k - |L| \cdot polylog(n) - (|N[x] - (|S_{low})| - |L|)$$
    $$ \geq \Omega(|S_{low}|)$$

    We know that total numner of edges indident to $S_{low}$ is $O(k|S_{low}|)$, each vertex in $F'$ will take at least $|S_{low}|$ edges, so $$ |F'| = O(\frac{k \cdot |S_{low}|}{ |S_{low}|}) = O(k)$$
    This proves claim 8.

    By claim 7 and 8, we have $|E1 \cup E2| = O(k|L|)$.

    For $E3$, the numner of edges incident to $x$ is bounded by $N[x] \leq 2k$. For edges indident to $t_x$, it must be incident to some edge in $E1 \cup E2$, or it would be removed in removal three. So it is also bounded by $O(k|L|)$. We proved lemma 1. 

  \end{subsection}

  \begin{subsection}{Algorithm aspect of Case 3}
    Building kernel can be done in sublinear time. We won't cover it in this report as it requires some more technical details in data structures and linear algebra. 

    Let's now go back to reality, where we don't have assumption 1. It turns out that we can we don't need the exact value of $k$ and $|L|$ for the kernel to be constructed successfully. If we have $\kappa$ and $\ell$, we only need to make sure $\kappa \geq k$ and $\frac{|L|}{2} \leq \ell \leq |L|$, then it will work. Otherwise, it will either return no cut or a cut that is too large. This results in the algorithm for case 3 assume we know $\kappa$:
    \begin{algorithm}
      \caption{Case 3}
      \begin{algorithmic}
        \State Let $X$ be the set of sampled verticies
        \For {each $x \in X$}
          \For {$i$ in 1 to $log(n)$}
            \State Let $\ell = 2^i$ 
            \State Sample $T$ with $\ell, \kappa$ 
            \State Compute kernel and run maxflow
            \State Repeat $log(n)$ times
          \EndFor
        \EndFor
        \State Return the smallest cut found
      \end{algorithmic}
    \end{algorithm}
  \end{subsection}

  \begin{section}{Putting things together}
    \begin{algorithm}
      \caption{Vertex connectivity}
    \begin{algorithmic}
    \State Let $k = \frac{n}{2}$
    \State Compute $k$-connectivity certificate for $G$
    \State Do case 1, 2, 3
    \State Verify and return the smallest cut found
    \State Binary search for $k$ and return the smallest cut found
    \end{algorithmic}
  \end{algorithm}
  \end{section}
  
\end{section}

\bibliographystyle{acm}
\bibliography{ref}

\includepdf[pages=-]{outline.pdf}



\end{document}